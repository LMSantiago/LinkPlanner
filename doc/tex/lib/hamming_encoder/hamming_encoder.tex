\clearpage

\section{Hamming Encoder}

\begin{tcolorbox}	
\begin{tabular}{p{2.75cm} p{0.2cm} p{10.5cm}} 	
\textbf{Header File}   &:& hamming\_coder\_20180806.h \\
\textbf{Source File}   &:& hamming\_coder\_20180806.cpp \\
\textbf{Version}       &:& 20180806
\end{tabular}
\end{tcolorbox}

\subsection*{Input Parameters}

This block accepts a single input parameter ($parityBits$). This variable defines the Hamming Algorithm used according to the table below.

\begin{table}[h!]
	\centering
	\begin{tabular}{|c|c|c|c|c|}
		\hline
		\textbf{\begin{tabular}[c]{@{}c@{}}Parity\\ Bits\end{tabular}} & \textbf{\begin{tabular}[c]{@{}c@{}}Total\\ Bits\end{tabular}} & \textbf{\begin{tabular}[c]{@{}c@{}}Data\\ Bits\end{tabular}} & \textbf{\begin{tabular}[c]{@{}c@{}}Hamming\\ Code\end{tabular}} & \textbf{Rate} \\ \hline
		\textbf{2} & 3 & 1 & (3, 1) & 1/3 \\ \hline
		\textbf{3} & 7 & 4 & (7, 4) & 4/7 \\ \hline
		\textbf{4} & 15 & 11 & (15, 11) & 11/15 \\ \hline
		\textbf{5} & 31 & 26 & (31, 26) & 26/31 \\ \hline
		\textbf{6} & 63 & 57 & (63, 57) & 57/63 \\ \hline
		\textbf{7} & 127 & 120 & (127, 120) & 120/127 \\ \hline
		\textbf{8} & 255 & 247 & (255, 247) & 247/255 \\ \hline
	\end{tabular}
\end{table}

\subsection*{Functional Description}

This block performs the encoding of the input signal using the Hamming Algorithm and outputs the encoded signal.

\subsection*{Input Signals}

\textbf{Number}: 1\\
\textbf{Type}: Real Signal (DiscreteTimeDiscreteAmplitude)

\subsection*{Output Signals}

\textbf{Number}: 1\\
\textbf{Type}: Real Signal (DiscreteTimeDiscreteAmplitude)

%\end{document}
