\clearpage

\section{Huffman Dynamic Coder}

\begin{tcolorbox}	
\begin{tabular}{p{2.75cm} p{0.2cm} p{10.5cm}} 	
\textbf{Header File}   &:& dynhuffcod.h \\
\textbf{Source File}   &:& dynhuffcod.cpp \\
\textbf{Version}       &:& 20180629
\end{tabular}
\end{tcolorbox}

\subsection*{Input Parameters}

This block does not have input parameters.

\subsection*{Functional Description}

This block implements a dynamic huffman coder. This encoder is used to compress text files, attributing shorter codes to characters that appear more frequently in the text.\\
It creates and updates a tree that store the codes used for every character. Each character is represented by a node, that has four fields: The character itself; Its frequency in the text; A pointer to the left soon character. A pointer to the right soon character. \\
When a new character is received two possible actions can be performed: It can be added to the tree, in case it did not exists before; Or it's frequency can be incremented, in case it is already part of the tree. \\
After each modification, the tree it is ordered following huffman algorithm.\\
The output code is the location of the character, if it exists, otherwise, it is the location of the empty tree leaf plus the character ASCII code $($8 bits$)$.

\subsection*{Input Signals}

\textbf{Number}: 1\\
\textbf{Type}: Binary $($considered as group of characters/bytes$)$

\subsection*{Output Signals}

\textbf{Number}: 1\\
\textbf{Type}: Binary $($huffman code of input text$)$

%\end{document}
