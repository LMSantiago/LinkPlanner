\clearpage

\section{Resample}

\begin{tcolorbox}	
	\begin{tabular}{p{2.75cm} p{0.2cm} p{10.5cm}} 	
		\textbf{Header File}   &:& resample$\_*$.h \\
		\textbf{Source File}   &:& resample$\_*$.cpp \\
        \textbf{Version}       &:& 20180423 (Celestino Martins) \\
	\end{tabular}
\end{tcolorbox}

This block simulates the resampling of a signal. It receives one input signal and outputs a signal with the sampling rate defined by sampling rate, which is externally configured.

\subsection*{Input Parameters}

\begin{table}[h]
	\centering
	\begin{tabular}{|c|c|c|c|cccc}
		\cline{1-4}
		\textbf{Parameter} & \textbf{Type} & \textbf{Values} &   \textbf{Default}& \\ \cline{1-4}
		rFactor           & double & any & $inf$ \\ \cline{1-4}
		samplingPeriod    & double & any & $0.0$ \\ \cline{1-4}	
        symbolPeriod      & double & any & $1.5$ \\ \cline{1-4}	
	\end{tabular}
	\caption{Resample input parameters}
	\label{table:resample_in_par}
\end{table}


\subsection*{Methods}

Resample() {};
Resample(vector$<$Signal *$>$ \&InputSig, vector$<$Signal *$>$ \&OutputSig) :Block(InputSig, OutputSig)\{\};

void initialize(void);
bool runBlock(void);

void setSamplingPeriod(double sPeriod) { samplingPeriod = sPeriod; }
void setSymbolPeriod(double sPeriod) { symbolPeriod = sPeriod; }

void setOutRateFactor(double OUTsRate) { rFactor = OUTsRate; }
double getOutRateFactor() { return rFactor; }

\subsection*{Functional description}

This block can performs the signal resample according to the defined input parameter \textit{rFactor}. It resamples the input signal at rFactor times the original sample rate.

Firstly, the parameter \textit{nBits} is checked and if it is greater than 1 it is performed a linear interpolation, increasing the input signal original sample rate to rFactor times.


\pagebreak
\subsection*{Input Signals}

\subparagraph*{Number:} 1

\subsection*{Output Signals}

\subparagraph*{Number:} 1

\subparagraph*{Type:} Electrical complex signal

\subsection*{Examples}

\subsection*{Sugestions for future improvement}


