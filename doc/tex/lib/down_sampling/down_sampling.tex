\clearpage

\section{DownSampling}

\begin{tcolorbox}	
	\begin{tabular}{p{2.75cm} p{0.2cm} p{10.5cm}} 	
		\textbf{Header File}   &:& down\_sampling$\_*$.h \\
		\textbf{Source File}   &:& down\_sampling$\_*$.cpp \\
        \textbf{Version}       &:& 20180917 (Celestino Martins) \\
	\end{tabular}
\end{tcolorbox}

This block simulates the down-sampling function, where the signal sample rate is decreased by integer factor.

\subsection*{Input Parameters}

\begin{table}[h]
	\centering
	\begin{tabular}{|c|c|c|c|cccc}
		\cline{1-4}
		\textbf{Parameter} & \textbf{Type} & \textbf{Values} &   \textbf{Default}& \\ \cline{1-4}
		downSamplingFactor & int & any & $2$ \\ \cline{1-4}	
	\end{tabular}
	\caption{DownSampling input parameters}
	\label{table:down_sampling}
\end{table}


\subsection*{Methods}

DownSampling() {};
\bigbreak
DownSampling(vector$<$Signal *$>$ \&InputSig, vector$<$Signal *$>$ \&OutputSig) :Block(InputSig, OutputSig)\{\};
\bigbreak
void initialize(void);
\bigbreak
bool runBlock(void);
\bigbreak
void setSamplingFactor(unsigned int dSamplingfactor) { downSamplingFactor = dSamplingfactor; }
\bigbreak
unsigned int getSamplingFactor() { return downSamplingFactor; }

\subsection*{Functional description}

This block perform decreases the sample rate of input signal by a factor of $downSamplingFactor$. Given a down-sampling factor, $downSamplingFactor$, the output signal correspond to the first sample of input signal and every $downSamplingFactor^{th}$ sample after the first.


\pagebreak
\subsection*{Input Signals}

\subparagraph*{Number:} 1

\subsection*{Output Signals}

\subparagraph*{Number:} 1

\subparagraph*{Type:} Electrical real signal

\subsection*{Examples}

\subsection*{Sugestions for future improvement}


