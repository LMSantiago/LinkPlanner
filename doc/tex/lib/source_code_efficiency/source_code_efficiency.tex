\clearpage

\section{Source Code Efficiency}

\begin{tcolorbox}	
\begin{tabular}{p{2.75cm} p{0.2cm} p{10.5cm}} 	
\textbf{Header File}   &:& source\_code\_efficiency\_*.h \\
\textbf{Source File}   &:&  source\_code\_efficiency\_*.cpp \\
\textbf{Version}       &:& 20180621 (MarinaJordao)
\end{tabular}
\end{tcolorbox}


\subsection*{Input Parameters}

This block accepts one input signal and it produces one output signal. To perform this block, two input variables are required, probabilityOfZero and sourceOrde.


\begin{table}[h]
	\centering
	\begin{tabular}{|c|c|p{60mm}|c|ccp{60mm}}
		\cline{1-4}
		\textbf{Parameter} & \textbf{Type} & \textbf{Values} &   \textbf{Default}& \\ \cline{1-4}
		probabilityOfZero & double & from 1 to 0 & $0.45$ \\ \cline{1-4}
		sourceOrder & int & 2, 3 or 4 & $2$ \\ \cline{1-4}
	\end{tabular}
	\caption{Source Code Efficiency input parameters}
	\label{table:sink_in_par}
\end{table}


\subsection*{Functional Description}

This block estimates the efficiency of the message, by calculating the entropy and the length of the message.

\subsection*{Input Signals}

\textbf{Number}: 1\\
\textbf{Type}: Binary

\subsection*{Output Signals}

\textbf{Number}: 1\\
\textbf{Type}: Real   (TimeContinuousAmplitudeContinuousReal)

%\end{document}
