\clearpage

\section{Huffman Dynamic Decoder}

\begin{tcolorbox}	
\begin{tabular}{p{2.75cm} p{0.2cm} p{10.5cm}} 	
\textbf{Header File}   &:& dynhuffdec.h \\
\textbf{Source File}   &:& dynhuffdec.cpp \\
\textbf{Version}       &:& 20180629
\end{tabular}
\end{tcolorbox}

\subsection*{Input Parameters}

This block does not have input parameters.

\subsection*{Functional Description}

This block implements a dynamic huffman decoder. It build a tree that matches exactly the one build in the coder so it can decode the dynamically encoded bytes.\\
If the input code leads to the empty leaf of the tree the next 8 bits represent the ASCII code of the character that is to be added to the tree. Otherwise the code leads to the leaf that stores the encoded character.

\subsection*{Input Signals}

\textbf{Number}: 1\\
\textbf{Type}: Binary $($huffman code of text$)$

\subsection*{Output Signals}

\textbf{Number}: 1\\
\textbf{Type}: Binary $($considered as group of characters/bytes$)$

%\end{document}